\documentclass[a4paper,11pt]{article}
\usepackage[utf8]{inputenc}
\usepackage[frenchb]{babel}
\usepackage[T1]{fontenc}
\usepackage{lmodern}
\usepackage[margin=28mm,bindingoffset=10mm]{geometry}
\usepackage{amsmath, amssymb,amsfonts}
\usepackage{mathtools}

%\addto\captionsfrench{\renewcommand{\partname}{\huge}}
%\renewcommand{\thepart}{\arabic{part}}
\renewcommand{\thesection}{\alph{section})}

\begin{document}

\part{Semi-group estimates}
Soit \( \sigma \in \mathbb{R} \) un coefficient donné. Consider the linear wave system in dimension one

Consideron le système d'onde linéaire en

\renewcommand{\theequation}{\Alph{equation}}

\begin{alignat}{1}
    \partial_{t}p + \partial_{x}v &= \sigma(v-p) \label{A} \\
    \partial_{t}v + \partial_{x}p &= \sigma(p-v) \label{B}
\end{alignat}

\setcounter{equation}{0}
\renewcommand{\theequation}{\arabic{equation}}

\section{Determine une répresentation explicite de \( U(x, t) \) en fonction des données initiales.}
\subsection*{(\ref{A})+(\ref{B})}

\begin{equation*}
    \begin{array}{rcl}
        \partial_{t}(p+v) + \partial_{x}(p+v) &=& \sigma(v-p) + \sigma(p-v) \\
        &=& \sigma(v-p + p-v) \\
        &=& \sigma(0) \\
        \partial_{t}(p+v) + \partial_{x}(p+v) &=& 0
    \end{array}
\end{equation*}

\subsection*{(\ref{A})-(\ref{B})}

\begin{equation*}
\begin{array}{rcl}
    \partial_{t}(p-v) - \partial_{x}(p-v) &=& \sigma(v-p) - \sigma(p-v) \\
    &=& \sigma(v-p - p+v) \\
    &=& \sigma(2v-2p) \\
    &=& 2\sigma(v-p) \\
    \partial_{t}(p-v) - \partial_{x}(p-v) &=& -2\sigma(p-v)
\end{array}
\end{equation*}

\subsection*{On a obtenu deux équations de transport :}
\begin{alignat}{1}
    \partial_{t}(p+v) + \partial_{x}(p+v) &= 0 \label{eq1} \\
    \partial_{t}(p-v) - \partial_{x}(p-v) &= -2\sigma(p-v) \label{eq2}
\end{alignat}

\subsection*{Résolution de (\ref{eq1}) par la méthode des caractéristiques.}

\begin{equation*}
\left\{
\begin{array}{rcl}
    \frac{\mathrm{d}x^{*}(t)}{\mathrm{d}t} &=& 1 \\
    x^{*}(t_{*}) &=& x_{*}
\end{array}
\right.
\end{equation*}

\begin{equation*}
\Rightarrow
\begin{array}{rcl}
x^{*}(t) &=& t +c \\
x^{*}(t_{*}) &=& t_{*}+c = x_{*}\\
c &=& x_{*}-t_{*} \\
x^{*}(t) &=& t + x_{*}-t_{*}
\end{array}
\end{equation*}

\begin{equation*}
\begin{array}{rcl}
    (p+v)(x_{*},t_{*}) &=& (p_{0}+v_{0})(x_{*},t_{*}) \\
    (p+v)(x,t) &=& p_{0}(x,t) + v_{0}(x,t)\\
\end{array}
\end{equation*}

\subsection*{Résolution de (\ref{eq2}) par la méthode des caractéristiques.}

\begin{equation*}
\left\{
\begin{array}{rcl}
    \frac{\mathrm{d}x_{1}^{*}(t)}{\mathrm{d}t} &=& -1 \\
    x_{1}^{*}(t_{*}) &=& x_{*}
\end{array}
\right. \\
\Rightarrow
\end{equation*}

\begin{equation*}
\begin{array}{rcl}
    x_{1}^{*}(t) &=& -t +c_{1} \\
    x_{1}^{*}(t_{*}) &=& x_{*} = -t_{*}+c_{1} \\
    c_{1} &=& x_{*} + t_{*} \\
    x^{*}(t) &=& -t + x_{*} + t_{*}
\end{array}
\end{equation*}

\begin{equation*}
\begin{array}{rcl}
    \frac{\mathrm{d}}{\mathrm{d}t}(p-v)(x_{1}^{*}(t), t) &=& -2\sigma (p-v)(x_{*}(t), t) \\
    (p-v)(x_{1}^{*}(t), t) &=& K \exp^{-2\sigma t} \\
    K &=& K \exp^{-2\sigma t} \\
    &=& (p-v)(x_{1}^{*}(0), 0) \\
    &=& (p_{0}-v_{0})(x_{1}^{*} + t_{*})
\end{array}
\end{equation*}

\begin{equation*}
\begin{array}{rcl}
    (p-v)(x_{1}^{*}(t), t) &=& (p_{0}-v_{0})(x_{*} + t_{*})\exp^{-2\sigma t_{*}} \\
    (p-v)(x,t) &=& (p_{0}-v_{0})(x,t)\exp^{-2\sigma t}
\end{array}
\end{equation*}

\subsection*{Solutions de \ref{eq1} et \ref{eq2}}

\begin{equation*}
\left\{
\begin{array}{rcl}
    (p+v) &=& (x - t) + u_{0}(x - t) \\
    (p-v) &=& (x + t) \exp^{-2\sigma t} - u_{0}(x + t) \exp^{-2\sigma t}
\end{array}
\right.
\end{equation*}

\subsection*{}

\begin{equation}
\begin{array}{rcll}
    u(x,t) &=& (p(x,t), v(x,t)) & \forall t \ge 0 \\
    avec & & & \\
    p(x,t) &=& 1/2\left[p_{0}(x - t) + u_{0}(x - t) + (p_{0}(x + t) - u_{0}(x + t))\exp^{-2\sigma t} \right] & \\
    v(x,t) &=& 1/2\left[p_{0}(x - t) + u_{0}(x - t) - (p_{0}(x + t) - u_{0}(x + t))\exp^{-2\sigma t} \right] &
\end{array}
\end{equation}

\subsection*{L'operator A, tel que \( u = \exp^{tA}u_{0} \)}

\begin{equation*}
\begin{array}{rcl}
    u(x,t) &=& (p(x,t), v(x,t)) \\
    \partial_{t}u(x,t) &=& (\partial_{t}p(x,t),  \partial_{t}v(x,t)) \\
    \partial_{x}u(x,t) &=& (\partial_{x}p(x,t),  \partial_{x}v(x,t)) \\
\end{array}
\end{equation*}

\begin{equation*}
\begin{array}{rcl}
    \partial_{x}
    \begin{pmatrix}
        v \\
        p
    \end{pmatrix} &=&
    \begin{pmatrix}
        0 & 1 \\
        1 & 0
    \end{pmatrix}
    \begin{pmatrix}
        \partial_{x}p \\
        \partial_{x}v
    \end{pmatrix}
\end{array}
\end{equation*}

\begin{equation*}
\begin{array}{rcl}
    \partial_{x}
    \begin{pmatrix}
        v \\
        p
    \end{pmatrix} &=&
    A_{1}\partial_{x}u, \quad avec \\
    A_{1} &=&
        \begin{pmatrix}
            0 & 1 \\
            1 & 0
        \end{pmatrix}
\end{array}
\end{equation*}

\begin{equation*}
\begin{array}{rcl}
    \begin{pmatrix}
        \sigma(v-p) \\
        \sigma(p-v)
    \end{pmatrix} &=&
    \begin{pmatrix}
        -\sigma & \sigma \\
        \sigma & -\sigma
    \end{pmatrix}
    \begin{pmatrix}
        p \\
        v
    \end{pmatrix}
\end{array}
\end{equation*}

\begin{equation*}
\begin{array}{rcl}
    \begin{pmatrix}
        \sigma(v-p) \\
        \sigma(p-v)
    \end{pmatrix} &=&
    Bu, \quad avec \\
    B &=&
    \begin{pmatrix}
        -\sigma & \sigma \\
        \sigma & -\sigma
    \end{pmatrix}
\end{array}
\end{equation*}

\section{}

\part{Numerical methods}

\section{}

\end{document}
