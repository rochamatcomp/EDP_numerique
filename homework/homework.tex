% !TeX encoding = UTF-8
% !TeX program = pdflatex
% !TeX spellcheck = fr_FR
\documentclass[a4paper,11pt]{article}
\usepackage[utf8]{inputenc}       % encodage à privilégier pour la portabilité et +
\usepackage[frenchb]{babel}       % francisation de libellés et de la typographie
\usepackage[T1]{fontenc}          % encodage européen des caractères (Cork)
\usepackage{lmodern}              % police europ'eennes vectorielles CM-like
\usepackage[margin=28mm,bindingoffset=10mm]{geometry}
\usepackage{amsmath, mathtools}   % équations, matrices, etc..
\usepackage{amssymb,amsfonts}  % tous les symboles math de AMS

\renewcommand{\thesubsection}{\alph{subsection})}

\begin{document}

\section{Semi-group estimates}
Pour \( \sigma \in \mathbb{R} \)
Considerons le système d'onde linéaire en dimension un

\begin{alignat}{1}
    \partial_{t}p + \partial_{x}v &= \sigma(v-p) \\
    \partial_{t}v + \partial_{x}p &= \sigma(p-v)
\end{alignat}

\subsection{Determine \( u = (p, v) \) en fonction de \( u_{0} = (p_{0}, v_{0}) \)}
(1)+(2)

\begin{equation*}
    \begin{array}{rcl}
        \partial_{t}(p+v) + \partial_{x}(p+v) &=& \sigma(v-p) + \sigma(p-v) \\
        &=& \sigma(v-p + p-v) \\
        &=& \sigma(0) \\
        \partial_{t}(p+v) + \partial_{x}(p+v) &=& 0
    \end{array}
\end{equation*}

(1)-(2)

\begin{equation*}
\begin{array}{rcl}
    ???\partial_{t}(p-v) + \partial_{x}(v-p) &=& \sigma(v-p) - \sigma(p-v) \\
    \partial_{t}(p-v) - \partial_{x}(p-v) &=& \sigma(v-p - p+v) \\
    &=& \sigma(2v-2p) \\
    &=& 2\sigma(v-p) \\
    \partial_{t}(p-v) - \partial_{x}(p-v) &=& -2\sigma(p-v)
\end{array}
\end{equation*}

\begin{alignat}{1}
    \partial_{t}(p+v) + \partial_{x}(p+v) &= 0 \\
    \partial_{t}(p-v) - \partial_{x}(p-v) &= -2\sigma(p-v)
\end{alignat}

On a deux équations de transport.

\subsubsection*{Resolution par la méthode des caractéristiques}
(3)

\begin{equation*}
\left\{
\begin{array}{rcl}
    \frac{dx^{*}(t)}{dt} &=& 1 \\
    x^{*}(t^{*}) &=& x_{*}
\end{array}
\right. \\
\Rightarrow
\end{equation*}

\begin{equation*}
\begin{array}{rcl}
x^{*}(t) &=& t +c \\
x^{*}(t^{*}) &=& x_{*} = t^{*}+c \\
c &=& x_{*}-t^{*} \\
x^{*}(t) &=& t + x_{*}-t^{*}
\end{array}
\end{equation*}

\begin{equation*}
\begin{array}{rcl}
    (p+v)(x_{*},t^{*}) &=& (p_{0}+v_{0})(x_{*},t^{*}) \\
    (p+v)(x,t) &=& p_{0}(x,t) + v_{0}(x,t)\\
\end{array}
\end{equation*}

(4)

\begin{equation*}
\left\{
\begin{array}{rcl}
    \frac{dx_{1}^{*}(t)}{dt} &=& -1 \\
    x_{1}^{*}(t^{*}) &=& x_{*}
\end{array}
\right. \\
\Rightarrow
\end{equation*}

\begin{equation*}
\begin{array}{rcl}
    x_{1}^{*}(t) &=& -t +c_{1} \\
    x_{1}^{*}(t^{*}) &=& x_{*} = -t^{*}+c_{1} \\
    c_{1} &=& x_{*} + t^{*} \\
    x^{*}(t) &=& -t + x_{*} + t^{*}
\end{array}
\end{equation*}

\begin{equation*}
\begin{array}{rcl}
    \frac{d}{dt}(p-v)(x_{1}^{*}(t), t) &=& -2\sigma (p-v)(x_{*}(t), t) \\
    (p-v)(x_{1}^{*}(t), t) &=& K \exp^{-2\sigma t} \\
    K &=& K \exp^{-2\sigma t} \\
    &=& (p-v)(x_{1}^{*}(0), 0) \\
    &=& (p_{0}-v_{0})(x_{1}^{*} + t^{*})
\end{array}
\end{equation*}

\begin{equation*}
\begin{array}{rcl}
    (p-v)(x_{1}^{*}(t), t) &=& (p_{0}-v_{0})(x_{*} + t^{*})\exp^{-2\sigma t^{*}} \\
    (p-v)(x,t) &=& (p_{0}-v_{0})(x,t)\exp^{-2\sigma t}
\end{array}
\end{equation*}

\begin{equation*}
\left\{
\begin{array}{rcl}
    (p+v) &=& (x - t) + u_{0}(x - t) \\
    (p-v) &=& (x + t) \exp^{-2\sigma t} - u_{0}(x + t) \exp^{-2\sigma t}
\end{array}
\right.
\end{equation*}

\begin{equation*}
\begin{array}{rcll}
    u(x,t) &=& (p(x,t), v(x,t)) & \forall t \ge 0 \\
    avec & & & \\
    p(x,t) &=& 1/2\left[p_{0}(x - t) + u_{0}(x - t) + (p_{0}(x + t) - u_{0}(x + t))\exp^{-2\sigma t} \right] & \\
    v(x,t) &=& 1/2\left[p_{0}(x - t) + u_{0}(x - t) - (p_{0}(x + t) - u_{0}(x + t))\exp^{-2\sigma t} \right] &
\end{array}
\end{equation*}


\subsubsection*{Écrire explicitement l'operator A, tel que \( u = \exp^{tA}u_{0} \)}

\begin{equation*}
\begin{array}{rcl}
    u &=& (p, v) \\
    \partial_{t}u &=& (\partial_{t}p,  \partial_{t}v) \\
    \partial_{x}u &=& (\partial_{x}p,  \partial_{x}v)
\end{array}
\end{equation*}

\begin{equation*}
\begin{array}{rclcl}
    \partial_{x}
    \begin{pmatrix}
        v \\
        p
    \end{pmatrix} &=&
    \begin{pmatrix}
        0 & 1 \\
        1 & 0
    \end{pmatrix}
    \begin{pmatrix}
        \partial_{x}p \\
        \partial_{x}v
    \end{pmatrix} &=&
    A_{1}\partial_{x}u, \quad avec \\
    A_{1} &=&
    \begin{pmatrix}
        0 & 1 \\
        1 & 0
    \end{pmatrix} & &
\end{array}
\end{equation*}

\begin{equation*}
\begin{array}{rclcl}
    \begin{pmatrix}
        \sigma(v-p) \\
        \sigma(p-v)
    \end{pmatrix} &=&
    \begin{pmatrix}
        -\sigma & \sigma \\
        \sigma & -\sigma
    \end{pmatrix}
    \begin{pmatrix}
        p \\
        v
    \end{pmatrix} &=&
    Bu, \quad avec \\
    B &=&
    \begin{pmatrix}
        -\sigma & \sigma \\
        \sigma & -\sigma
    \end{pmatrix} & &
\end{array}
\end{equation*}



\section{Numerical methods}

\begin{equation}
    \left\{
    \begin{array}{rclll}
        \partial_{t}u - \partial_{xx}u &=& 0, &x \in \mathbb{R}, & t>0, \\
        u(0, x) &=& u_{0}(x), &x \in \mathbb{R} &
    \end{array}
    \right.
\end{equation}

La discrétisation de type Différences Finis explicite avec un schéma sur la forme :
\begin{equation}
    \begin{array}{rcl}
        \frac{u^{n+1}_{j} - u^{n}_{j}}{\Delta t}
        -\frac{4}{3} \frac{u^{n}_{j+1} - 2u^{n}_{j} + u^{n}_{j-1}}{\Delta x^{2}}
        +\frac{1}{12} \frac{u^{n}_{j+2} - 2u^{n}_{j} + u^{n}_{j-2}}{\Delta x^{2}}
        -\frac{\Delta t^{2}}{2} \frac{u^{n}_{j+2} - 4u^{n}_{j+1} + 6u^{n}_{j}  - 4u^{n}_{j-1} + u^{n}_{j-2}}{\Delta x^{4}}  &=& 0 \label{scheme}
    \end{array}
\end{equation}


\subsection{Determination du symbol du schéma}

Le symbole du schéma (\ref{scheme}) est donné par :
\begin{equation*}
\lambda(\theta) = \sum \limits_{r=-2}^{2} \alpha_{r} e^{\mathbf{i} \theta r}, \quad \theta \in \mathbb{R}
\end{equation*}

Par développement de ce schéma on a :

\begin{equation*}
    \begin{array}{rcl}
        u^{n+1}_{j} &=& \alpha_{0} u^{n}_{j} + \alpha_{-1} u^{n}_{j-1} + \alpha_{-2} u^{n}_{j-2} + \alpha_{1} u^{n}_{j+1}  + \alpha_{2} u^{n}_{j+2}, \quad o\grave{u}
    \end{array}
\end{equation*}

\begin{equation*}
    \left\{
    \begin{array}{rcl}
        \alpha_{0} &=& 1 - \scriptstyle \frac{15}{6} \nu + 3 \nu^{2} \\
        \alpha_{-1} &=& \scriptstyle \frac{4}{3} \nu - 2 \nu^{2} \\
        \alpha_{-2} &=& \scriptstyle \frac{-1}{12} \nu + \frac{1}{2} \nu^{2} \\
        \alpha_{1} &=& \scriptstyle \frac{4}{3} \nu - 2 \nu^{2} \\
        \alpha_{2} &=& \scriptstyle \frac{-1}{12} \nu + \frac{1}{2} \nu^{2}
    \end{array}
    \right.,
\end{equation*}

\begin{equation*}
    avec \quad
    \begin{array}{rcl}
        \nu &=& \Delta t/\Delta x^{2} \\
        \nu^{2} &=& \Delta t^{2}/\Delta x^{4}
    \end{array}
\end{equation*}


\section{Consistence du schéma}

Par développement de Taylor du schéma (\ref{scheme}) on a:

\begin{equation*}
    u(x_{j-1}, t^{n}) =
    u(x_{j}, t^{n})
     - \Delta x \frac{\partial}{\partial x}u(x_{j}, t^{n})
     + \frac{(\Delta x)^{2}}{2} \frac{\partial^{2}}{\partial x^{2}}u(x_{j}, t^{n})
     - \frac{(\Delta x)^{3}}{3!} \frac{\partial^{3}}{\partial x^{3}}u(x_{j}, t^{n})
     + \mathcal{O}((\Delta x)^{4})
\end{equation*}

\begin{equation*}
    u(x_{j+1}, t^{n}) =
    u(x_{j}, t^{n})
     - \Delta x \frac{\partial}{\partial x}u(x_{j}, t^{n})
     + \frac{(\Delta x)^{2}}{2} \frac{\partial^{2}}{\partial x^{2}}u(x_{j}, t^{n})
     - \frac{(\Delta x)^{3}}{3!} \frac{\partial^{3}}{\partial x^{3}}u(x_{j}, t^{n})
     + \mathcal{O}((\Delta x)^{4})
\end{equation*}

\begin{equation*}
    u(x_{j-2}, t^{n}) =
    u(x_{j}, t^{n})
     - 2\Delta x \frac{\partial}{\partial x}u(x_{j}, t^{n})
     + \frac{(2\Delta x)^{2}}{2} \frac{\partial^{2}}{\partial x^{2}}u(x_{j}, t^{n})
     - \frac{(2\Delta x)^{3}}{6} \frac{\partial^{3}}{\partial x^{3}}u(x_{j}, t^{n})
     + \mathcal{O}((\Delta x)^{4})
\end{equation*}

\begin{equation*}
    u(x_{j+2}, t^{n}) =
    u(x_{j}, t^{n})
     - 2\Delta x \frac{\partial}{\partial x}u(x_{j}, t^{n})
     + \frac{(2\Delta x)^{2}}{2} \frac{\partial^{2}}{\partial x^{2}}u(x_{j}, t^{n})
     - \frac{(2\Delta x)^{3}}{6} \frac{\partial^{3}}{\partial x^{3}}u(x_{j}, t^{n})
     + \mathcal{O}((\Delta x)^{4})
\end{equation*}

\begin{equation*}
    u(x_{j}, t^{n+1}) =
    u(x_{j}, t^{n})    
     + \Delta t \frac{\partial}{\partial t}u(x_{j}, t^{n})
     + \frac{(\Delta t)^{2}}{2} \frac{\partial^{2}}{\partial t^{2}}u(x_{j}, t^{n})
     + \mathcal{O}((\Delta t)^{3})
\end{equation*}

\end{document}
