% !TeX encoding = UTF-8
% !TeX program = pdflatex
% !TeX spellcheck = fr_FR
\documentclass[a4paper,11pt]{article}
\usepackage[utf8]{inputenc}       % encodage à privilégier pour la portabilité et +
\usepackage[frenchb]{babel}       % francisation de libellés et de la typographie
\usepackage[T1]{fontenc}          % encodage européen des caractères (Cork)
\usepackage{lmodern}              % police europ'eennes vectorielles CM-like
\usepackage[margin=28mm,bindingoffset=10mm]{geometry}
\usepackage{amsmath, mathtools}   % équations, matrices, etc..
\usepackage{amssymb,amsfonts}  % tous les symboles math de AMS

\renewcommand{\thesubsection}{\alph{subsection})}

\begin{document}

\section{Semi-group estimates}
Pour \( \sigma \in \mathbb{R} \)
Considerons le système d'onde linéaire en dimension un

\begin{alignat}{1}
    \partial_{t}p + \partial_{x}v &= \sigma(v-p) \\
    \partial_{t}v + \partial_{x}p &= \sigma(p-v)
\end{alignat}

\subsection{Determine \( u = (p, v) \) en fonction de \( u_{0} = (p_{0}, v_{0}) \)}
(1)+(2)

\begin{equation*}
    \begin{array}{rcl}
        \partial_{t}(p+v) + \partial_{x}(p+v) &=& \sigma(v-p) + \sigma(p-v) \\
        &=& \sigma(v-p + p-v) \\
        &=& \sigma(0) \\
        \partial_{t}(p+v) + \partial_{x}(p+v) &=& 0
    \end{array}
\end{equation*}

(1)-(2)

\begin{equation*}
\begin{array}{rcl}
    \partial_{t}(p-v) + \partial_{x}(v-p) &=& \sigma(v-p) - \sigma(p-v) \\
    \partial_{t}(p-v) - \partial_{x}(p-v) &=& \sigma(v-p - p+v) \\
    &=& \sigma(2v-2p) \\
    &=& 2\sigma(v-p) \\
    \partial_{t}(p-v) - \partial_{x}(p-v) &=& -2\sigma(p-v)
\end{array}
\end{equation*}

\begin{alignat}{1}
    \partial_{t}(p+v) + \partial_{x}(p+v) &= 0 \\
    \partial_{t}(p-v) - \partial_{x}(p-v) &= -2\sigma(p-v)
\end{alignat}

On a deux équations de transport.

\subsubsection*{Resolution par la méthode des caractéristiques}
Par l'équation (3) 
\begin{equation*}
\partial_{t}(p+v) + \partial_{x}(p+v) = 0 
\end{equation*}

\begin{equation*}
\left\{
\begin{array}{rcl}
    \frac{dx^{*}(t)}{dt} &=& 1 \\
    x^{*}(t^{*}) &=& x_{*}
\end{array}
\right. \\
\Rightarrow
\end{equation*}

\begin{equation*}
\begin{array}{rcl}
x^{*}(t) &=& t +c \\
x^{*}(t^{*}) &=& x_{*} = t^{*}+c \\
c &=& x_{*}-t^{*} \\
x^{*}(t) &=& t + x_{*}-t^{*}
\end{array}
\end{equation*}

\begin{equation*}
\begin{array}{rcl}
    (p+v)(x_{*},t^{*}) &=& (p_{0}+v_{0})(x_{*}-t^{*}) \\
    (p+v)(x,t) &=& p_{0}(x-t) + v_{0}(x-t) \\
    p(x,t) + v(x,t) &=& p_{0}(x-t) + v_{0}(x-t)
\end{array}
\end{equation*}

Par l'équation (4)
\begin{equation*}
\partial_{t}(p-v) - \partial_{x}(p-v) = -2\sigma(p-v)
\end{equation*}

\begin{equation*}
\left\{
\begin{array}{rcl}
    \frac{dx_{1}^{*}(t)}{dt} &=& -1 \\
    x_{1}^{*}(t^{*}) &=& x_{*}
\end{array}
\right. \\
\Rightarrow
\end{equation*}

\begin{equation*}
\begin{array}{rcl}
    x_{1}^{*}(t) &=& -t +c_{1} \\
    x_{1}^{*}(t^{*}) &=& x_{*} = -t^{*}+c_{1} \\
    c_{1} &=& x_{*} + t^{*} \\
    x^{*}(t) &=& -t + x_{*} + t^{*}
\end{array}
\end{equation*}

\begin{equation*}
\begin{array}{rcl}
    \frac{d}{dt}(p-v)(x_{1}^{*}(t), t) &=& -2\sigma (p-v)(x_{*}(t), t) \\
    (p-v)(x_{1}^{*}(t), t) &=& K \exp^{-2\sigma t} \\
    K &=& K \exp^{-2\sigma t} \\
    &=& (p-v)(x_{1}^{*}(0), 0) \\
    &=& (p_{0}-v_{0})(x_{1}^{*} + t^{*})
\end{array}
\end{equation*}

\begin{equation*}
\begin{array}{rcl}
    (p-v)(x_{1}^{*}(t), t) &=& (p_{0}-v_{0})(x_{*} + t^{*})\exp^{-2\sigma t^{*}} \\
    (p-v)(x,t) &=& (p_{0}-v_{0})(x+t)\exp^{-2\sigma t}
\end{array}
\end{equation*}

\begin{equation*}
\left\{
\begin{array}{rcl}
    p(x, t) + v(x, t) &=& p_{0}(x - t) + v_{0}(x - t) \\
    p(x, t) - v(x, t) &=& p_{0}(x + t) \exp^{-2\sigma t} - v_{0}(x + t) \exp^{-2\sigma t}
\end{array}
\right.
\end{equation*}

On va résoudre le système. \\ \\
Par somme on a :
\begin{equation*}
\begin{array}{rcll}
    p(x,t) &=& 1/2\left[p_{0}(x - t) + v_{0}(x - t) + (p_{0}(x + t) - v_{0}(x + t))\exp^{-2\sigma t} \right] &
\end{array}
\end{equation*}

Par différence on a : 
\begin{equation*}
\begin{array}{rcll}
    v(x,t) &=& 1/2\left[p_{0}(x - t) + v_{0}(x - t) - (p_{0}(x + t) - v_{0}(x + t))\exp^{-2\sigma t} \right] &
\end{array}
\end{equation*}

Donc \( \forall t > 0 \)
\begin{equation*}
\begin{array}{rcll}
    u(x,t) &=& (p(x,t), v(x,t)) &
\end{array}
\end{equation*}


\subsubsection*{Écrire explicitement l'operator A, tel que \( u = \exp^{tA}u_{0} \)}

\begin{equation*}
\begin{array}{rcl}
    u &=& (p, v) \\
    \partial_{t}u &=& (\partial_{t}p,  \partial_{t}v) \\
    \partial_{x}u &=& (\partial_{x}p,  \partial_{x}v)
\end{array}
\end{equation*}

\begin{equation*}
\begin{array}{rclcl}
    \partial_{x}
    \begin{pmatrix}
        p \\
        v
    \end{pmatrix} &=&
    \begin{pmatrix}
        0 & 1 \\
        1 & 0
    \end{pmatrix}
    \begin{pmatrix}
        \partial_{x}p \\
        \partial_{x}v
    \end{pmatrix} &=&
    A_{1}\partial_{x}u, \quad avec \\
    A_{1} &=&
    \begin{pmatrix}
        0 & 1 \\
        1 & 0
    \end{pmatrix} & &
\end{array}
\end{equation*}

\begin{equation*}
\begin{array}{rclcl}
    \begin{pmatrix}
        \sigma(v-p) \\
        \sigma(p-v)
    \end{pmatrix} &=&
    \begin{pmatrix}
        -\sigma & \sigma \\
        \sigma & -\sigma
    \end{pmatrix}
    \begin{pmatrix}
        p \\
        v
    \end{pmatrix} &=&
    Bu, \quad avec \\
    B &=&
    \begin{pmatrix}
        -\sigma & \sigma \\
        \sigma & -\sigma
    \end{pmatrix} & &
\end{array}
\end{equation*}

\begin{displaymath}
    \partial_t U = AU \,\,\,\,\,\, \mbox{si on pose}\,\,\,\,\, A=-A_1\partial_x + B
\end{displaymath}

\begin{displaymath}
    \left\{
    \begin{array}{rl}
        \partial_t U &=AU\\
        U_0 &=(p_0,v_0).
        \end{array}
    \right.
    \Rightarrow \\
    U(t) = e^{At} u_{0}
\end{displaymath}

\subsection{\( Y_{1} = L^{1}(\mathbb{R})^{2} \)}

\begin{equation*}
\begin{array}{rcl}
    ||(a,b)||_{1} &=& ||a||_{L^{1}(\mathbb{R})} + ||b||_{L^{1}(\mathbb{R})}
\end{array}
\end{equation*}

Montrons que :
\( ||e^{At}||_{\mathcal{L}(Y_p)}\leq (1+e^{-2\sigma t}) \qquad \mbox{pour tout} \qquad t\geq 0. \)\\ \\
On a \( U(x,t)=(p(x,t),v(x,t)) \) \\

\begin{equation*}
\begin{split}
    ||U(t)||_1 &= ||p(t)||_{L^1(\mathbb{R})} + ||v(t)||_{L^1(\mathbb{R})} \\
    &= \int_{\mathbb{R}}{|p(x,t)|\,dx} + \int_{\mathbb{R}}{|v(x,t)|\,dx} \\
    &= \frac{1}{2} \int_{\mathbb{R}}|p_0(x-t)+v_0(x-t) + [p_0(x+t)-v_0(x+t)] e^{-2\sigma t} | \,dx \\
    & +\frac{1}{2} \int_{\mathbb{R}}|p_0(x-t)+v_0(x-t) - [p_0(x+t)-v_0(x+t)] e^{-2\sigma t} | \,dx
\end{split}
\end{equation*}

\begin{equation*}
\begin{split}
    ||U(t)||_1 & \le \frac{1}{2} \left[\int_{\mathbb{R}}|p_0(x-t)| \,dx + \int_{\mathbb{R}} |v_0(x-t)| \,dx + \right. \\
   & \qquad \left. + e^{-2\sigma t} \left( \int_{\mathbb{R}} |p_0(x+t)| \,dx + \int_{\mathbb{R}} |v_0(x+t)| \,dx \right) \right] + \\
   & + \frac{1}{2} \left[ \int_{\mathbb{R}}|p_0(x-t)| \,dx + \int_{\mathbb{R}} |v_0(x-t)| \,dx + \right. \\
   & \qquad \left. + e^{-2\sigma t} \left( \int_{\mathbb{R}} |p_0(x+t)| \,dx + \int_{\mathbb{R}} |v_0(x+t)| \,dx \right) \right]
\end{split}
\end{equation*}

On fait un changement de variable à chaque terme d'integrale :
\begin{equation*}
\begin{split}
    ||U(t)||_1 & \le \frac{1}{2} \left[\int_{\mathbb{R}}|p_0(x)| \,dx + \int_{\mathbb{R}} |v_0(x)| \,dx + \right. \\
   & \qquad \left. + e^{-2\sigma t} \left(\int_{\mathbb{R}} |p_0(x)| \,dx + \int_{\mathbb{R}} |v_0(x)| \,dx \right) \right] + \\
   & + \frac{1}{2} \left[ \int_{\mathbb{R}}|p_0(x)| \,dx + \int_{\mathbb{R}} |v_0(x)| \,dx + \right. \\
   & \qquad \left. + e^{-2\sigma t} \left(\int_{\mathbb{R}} |p_0(x)| \,dx + \int_{\mathbb{R}} |v_0(x)| \,dx \right) \right]
\end{split}
\end{equation*}

\begin{equation*}
\begin{split}
    ||U(t)||_1 & \le \int_{\mathbb{R}}|p_0(x)| \,dx + \int_{\mathbb{R}} |v_0(x)| \,dx + \\
   & \qquad + e^{-2\sigma t} \left( \int_{\mathbb{R}} |p_0(x)| \,dx + \int_{\mathbb{R}} |v_0(x)| \,dx \right)
\end{split}
\end{equation*}

\begin{equation*}
\begin{split}
    ||U(t)||_1 & \le ||p_0||_{L^{1}} + ||v_0||_{L^{1}} 
    + e^{-2\sigma t} \left(||p_0||_{L^{1}} + ||v_0||_{L^{1}} \right)
\end{split}
\end{equation*}

\begin{equation*}
\begin{split}
    ||U(t)||_1 & \le ||U_0||_{1} 
    + e^{-2\sigma t} ||U_0||_{1}
\end{split}
\end{equation*}

\begin{equation}
    ||U(t)||_1 \le \left( 1 + e^{-2\sigma t} \right) ||U_0||_{1}
\end{equation}

\begin{equation*}
||e^{At}||_{\mathcal{L}(Y_p)} = \sup_{||U_{0}||_1 = 1}\frac{||U(t)||_1}{||U_{0}||_1}
\end{equation*}

\begin{equation*}
    ||e^{At}||_{\mathcal{L}(Y_p)} \le 1 + e^{-2\sigma t}
\end{equation*}

%%% L2

\subsection{\( Y_{2} = L^{2}(\mathbb{R})^{2} \)}

\begin{equation*}
\begin{array}{rcl}
    ||(a,b)||_{2} &=& \sqrt{||a||^{2}_{L^{2}(\mathbb{R})} + ||b||^{2}_{L^{2}(\mathbb{R})}}
\end{array}
\end{equation*}

Montrons que :
\( ||e^{At}||_{\mathcal{L}(Y_2)}\leq max(1, e^{-2\sigma t}) \qquad \mbox{pour tout} \qquad t\geq 0. \)\\ \\
On a \( U(x,t)=(p(x,t),v(x,t)) \) \\

\begin{equation*}
\begin{split}
    ||U(t)||^{2}_{2} &= ||p(t)||^{2}_{L^{2}(\mathbb{R})} + ||v(t)||^{2}_{L^{2}(\mathbb{R})} \\
    &= \int_{\mathbb{R}}{|p(x,t)|^{2}\,dx} + \int_{\mathbb{R}}{|v(x,t)|^{2}\,dx} \\
    &= \frac{1}{4} \int_{\mathbb{R}} |p_0(x-t)+v_0(x-t) + (p_0(x+t)-v_0(x+t)) e^{-2\sigma t} |^{2} \,dx \\
    & \quad +\frac{1}{4} \int_{\mathbb{R}} |p_0(x-t) + v_0(x-t) - (p_0(x+t) - v_0(x+t)) e^{-2\sigma t}|^{2} \,dx 
\end{split}
\end{equation*}

\begin{equation*}
\begin{split}
    ||U(t)||^{2}_{2} & \le \frac{1}{4} \left[ \int_{\mathbb{R}}|p_0(x-t) + v_0(x-t)|^{2} \,dx \right. \\
   & \quad \left. + \int_{\mathbb{R}} |p_0(x+t) - v_0(x+t)|^{2}  e^{-4\sigma t} \,dx \right. \\
   & \quad \left. + 2 \int_{\mathbb{R}} |p_0(x-t) + v_0(x-t)| |p_0(x+t) - v_0(x+t)| e^{-2\sigma t} \,dx \right] \\
   & \quad + \frac{1}{4} \left[ \int_{\mathbb{R}}|p_0(x-t) + v_0(x-t)|^{2} \,dx \right. \\
   & \quad \left. + \int_{\mathbb{R}} |p_0(x+t) - v_0(x+t)|^{2}  e^{-4\sigma t} \,dx \right. \\
   & \quad \left. - 2 \int_{\mathbb{R}} |p_0(x-t) + v_0(x-t)| |p_0(x+t) - v_0(x+t)| e^{-2\sigma t} \,dx \right]
\end{split}
\end{equation*}

\begin{equation*}
\begin{split}
    ||U(t)||^{2}_{2} & \le \frac{1}{2} \left[ \int_{\mathbb{R}}|p_0(x-t) + v_0(x-t)|^{2} \,dx \right. \\
   & \quad \left. + \int_{\mathbb{R}} |p_0(x+t) - v_0(x+t)|^{2}  e^{-4\sigma t} \,dx \right] \\
   & \le \frac{1}{2} \left[ max(1, e^{-4\sigma t}) 
   \int_{\mathbb{R}} |p_0(x-t) + v_0(x-t)|^{2} \,dx \right. \\
   & \left. \qquad + max(1, e^{-4\sigma t}) 
   \int_{\mathbb{R}} |p_0(x+t) - v_0(x+t)|^{2} \,dx \right] \\
   & \le \frac{1}{2} max(1, e^{-4\sigma t}) \left[ \int_{\mathbb{R}} |p_0(x-t) + v_0(x-t)|^{2} \,dx + \int_{\mathbb{R}} |p_0(x+t) - v_0(x+t)|^{2} \,dx \right]
\end{split}
\end{equation*}

\begin{equation*}
\begin{split}
    ||U(t)||^{2}_{2} & \le \frac{1}{2} max(1, e^{-4\sigma t}) \left[ \int_{\mathbb{R}}|p_0(x-t)|^{2} \,dx + \int_{\mathbb{R}} |v_0(x-t)|^{2} \,dx \right. \\
   & \quad \left. + 2 \int_{\mathbb{R}} |p_0(x-t)| |v_0(x-t)| \,dx \right. \\
   & \quad + \left. \int_{\mathbb{R}}|p_0(x+t)|^{2} \,dx + \int_{\mathbb{R}} |v_0(x+t)|^{2} \,dx \right. \\
   & \quad \left. - 2 \int_{\mathbb{R}} |p_0(x+t)| |v_0(x+t)| \,dx \right] \\
\end{split}
\end{equation*}

On fait un changement de variable à chaque terme d'integrale :
\begin{equation*}
\begin{split}
    ||U(t)||^{2}_{2} & \le \frac{1}{2} max(1, e^{-4\sigma t}) \left[ \int_{\mathbb{R}}|p_0(x)|^{2} \,dx + \int_{\mathbb{R}} |v_0(x)|^{2} \,dx \right. \\
   & \left. \qquad  + 2 \int_{\mathbb{R}} |p_0(x)| |v_0(x)| \,dx \right. \\
   & \left. \qquad + \int_{\mathbb{R}}|p_0(x)|^{2} \,dx + \int_{\mathbb{R}} |v_0(x)|^{2} \,dx \right. \\
   & \qquad \left. - 2 \int_{\mathbb{R}} |p_0(x)| |v_0(x)| \,dx 
    \right]
\end{split}
\end{equation*}


\begin{equation*}
\begin{split}
    ||U(t)||^{2}_{2} & \le max(1, e^{-4\sigma t}) \left[ \int_{\mathbb{R}}|p_0(x)|^{2} \,dx + \int_{\mathbb{R}} |v_0(x)|^{2} \,dx \right] \\
\end{split}
\end{equation*}


\begin{equation*}
    ||U(t)||^{2}_{2} \le max(1, e^{-4\sigma t}) ||U_{0}||^{2}_{2}
\end{equation*}

\begin{equation*}
    ||U(t)||_{2} \le max(1, e^{-2\sigma t}) ||U_{0}||_{2}
\end{equation*}

\begin{equation*}
||e^{At}||_{\mathcal{L}(Y_{2})} = \sup_{||U_{0}||_{2} = 1}\frac{||U(t)||_{2}}{||U_{0}||_{2}}
\end{equation*}

\begin{equation*}
    ||e^{At}||_{\mathcal{L}(Y_{2})} \le max(1, e^{-2\sigma t})
\end{equation*}

%%% L INFINIT

\subsection{\( Y_{\infty} = L^{\infty}(\mathbb{R})^{2} \)}

\begin{equation*}
\begin{array}{rcl}
    ||(a,b)||_{\infty} &=& max(||a||_{L^{\infty}(\mathbb{R})}, ||b||_{L^{\infty}(\mathbb{R})}) \\
    \sigma = 0, & et & t\ge 0
\end{array}
\end{equation*}

Montrons que $||e^{At}||_{\mathcal{L}(Y_{\infty})} \leq 2$.\\ \\
Par d\'efinition, nous avons :

\begin{displaymath}
    ||U(t)||_{\infty}=\max(||p(t)||_{L^{\infty}(\mathbb{R})},||v(t)||_{L^{\infty}(\mathbb{R})})
\end{displaymath}
avec
\begin{displaymath}
    ||p(t)||_{L^{\infty}(\mathbb{R})}=\sup_{x\in \mathbb{R}}|p(x,t)| \quad \mbox{et} \quad ||v(t)||_{L^{\infty}(\mathbb{R})}=\sup_{x\in \mathbb{R}}|v(x,t)|.
\end{displaymath}

\begin{displaymath}
    \begin{split}
        ||U(t)||_{\infty} &= \max \left( \frac{1}{2}\sup_{x \in \mathbb{R}}|p_0(x-t)| + \frac{1}{2}\sup_{x \in \mathbb{R}} |v_0(x-t)| + \frac{1}{2}\sup_{x \in \mathbb{R}}|p_0(x+t)| + \frac{1}{2}\sup_{x \in \mathbb{R}}|v_0(x+t)| \right. , \\
        & \qquad \left. \frac{1}{2}\sup_{x \in \mathbb{R}}|p_0(x-t)| + \frac{1}{2}\sup_{x \in \mathbb{R}}|v_0(x-t)| + \frac{1}{2}\sup_{x \in \mathbb{R}}|p_0(x+t)| + \frac{1}{2}\sup_{x \in \mathbb{R}}|v_0(x+t)| \right)
    \end{split}
\end{displaymath}


Par un changement de variable, on a :

\begin{displaymath}
    \begin{split}
        ||U(t)||_{\infty} &= \max \left( \frac{1}{2}\sup_{x \in \mathbb{R}}|p_0(x)| + \frac{1}{2}\sup_{x \in \mathbb{R}} |v_0(x)| + \frac{1}{2}\sup_{x \in \mathbb{R}}|p_0(x)| + \frac{1}{2}\sup_{x \in \mathbb{R}}|v_0(x)| \right. , \\
        & \qquad \left. \frac{1}{2}\sup_{x \in \mathbb{R}}|p_0(x)| + \frac{1}{2}\sup_{x \in \mathbb{R}}|v_0(x)| + \frac{1}{2}\sup_{x \in \mathbb{R}}|p_0(x)| + \frac{1}{2}\sup_{x \in \mathbb{R}}|v_0(x)| \right)
    \end{split}
\end{displaymath}

\begin{displaymath}
    \begin{split}
        ||U(t)||_{\infty} &= \max \left( \frac{1}{2}\sup_{x \in \mathbb{R}}|p_0(x)| + \frac{1}{2}\sup_{x \in \mathbb{R}} |v_0(x)| + \frac{1}{2}\sup_{x \in \mathbb{R}}|p_0(x)| + \frac{1}{2}\sup_{x \in \mathbb{R}}|v_0(x)| \right. , \\
        & \qquad \left. \frac{1}{2}\sup_{x \in \mathbb{R}}|p_0(x)| + \frac{1}{2}\sup_{x \in \mathbb{R}}|v_0(x)| + \frac{1}{2}\sup_{x \in \mathbb{R}}|p_0(x)| + \frac{1}{2}\sup_{x \in \mathbb{R}}|v_0(x)| \right)
    \end{split}
\end{displaymath}



ce qui donne :
\begin{equation}
||v(t)||_{L^{\infty}(\mathbb{R})} \leq (1+e^{-2\sigma t})\max(||p_0||_{L^{\infty}(\mathbb{R})},||v_0||_{L^{\infty}(\mathbb{R})}).
\label{16} 
\end{equation}
Par suite ~\eqref{15} et ~\eqref{16} implique que :
\begin{displaymath}
||U(t)||_{\infty} \leq (1+e^{-2\sigma t})||U_0||_{\infty}
\end{displaymath}
c'est \`a dire que :
\begin{displaymath}
||e^{At}U_0||_{\infty} \leq (1+e^{-2\sigma t})||U_0||_{\infty}
\end{displaymath}
d'o\`u
\begin{equation}
||e^{At}||_{\mathcal{L}(Y_{\infty})}\leq 1+e^{-2\sigma t}
\label{17} 
\end{equation}
puisque
\begin{displaymath}
||e^{At}||_{\mathcal{L}(Y_{\infty})}=\sup_{U_0\in Y_{\infty}, ||U_0||_{\infty} \leq 1}||e^{At}U_0||_{\infty}.
\end{displaymath}
Et ainsi, lorsque $\sigma=0$, on obtient :
\begin{equation}
||e^{At}||_{\mathcal{L}(Y_{\infty})}\leq 2.
\label{18} 
\end{equation} 















\section{Numerical methods}

\begin{equation}
    \left\{
    \begin{array}{rclll}
        \partial_{t}u - \partial_{xx}u &=& 0, &x \in \mathbb{R}, & t>0, \\
        u(0, x) &=& u_{0}(x), &x \in \mathbb{R} &
    \end{array}
    \right.
\end{equation}

La discrétisation de type Différences Finis explicite avec un schéma sur la forme :
\begin{equation}
    \begin{array}{rcl}
        \frac{u^{n+1}_{j} - u^{n}_{j}}{\Delta t}
        -\frac{4}{3} \frac{u^{n}_{j+1} - 2u^{n}_{j} + u^{n}_{j-1}}{\Delta x^{2}}
        +\frac{1}{12} \frac{u^{n}_{j+2} - 2u^{n}_{j} + u^{n}_{j-2}}{\Delta x^{2}}
        -\frac{\Delta t^{2}}{2} \frac{u^{n}_{j+2} - 4u^{n}_{j+1} + 6u^{n}_{j}  - 4u^{n}_{j-1} + u^{n}_{j-2}}{\Delta x^{4}}  &=& 0 \label{scheme}
    \end{array}
\end{equation}


\subsection{Determination du symbol du schéma}

Le symbole du schéma (\ref{scheme}) est donné par :
\begin{equation*}
\lambda(\theta) = \sum \limits_{r=-2}^{2} \alpha_{r} e^{\mathbf{i} \theta r}, \quad \theta \in \mathbb{R}
\end{equation*}

Par développement de ce schéma on a :

\begin{equation*}
    \begin{array}{rcl}
        u^{n+1}_{j} &=& \alpha_{0} u^{n}_{j} + \alpha_{-1} u^{n}_{j-1} + \alpha_{-2} u^{n}_{j-2} + \alpha_{1} u^{n}_{j+1}  + \alpha_{2} u^{n}_{j+2}, \quad o\grave{u}
    \end{array}
\end{equation*}

\begin{equation*}
    \left\{
    \begin{array}{rcl}
        \alpha_{0} &=& 1 - \scriptstyle \frac{15}{6} \nu + 3 \nu^{2} \\
        \alpha_{-1} &=& \scriptstyle \frac{4}{3} \nu - 2 \nu^{2} \\
        \alpha_{-2} &=& \scriptstyle \frac{-1}{12} \nu + \frac{1}{2} \nu^{2} \\
        \alpha_{1} &=& \scriptstyle \frac{4}{3} \nu - 2 \nu^{2} \\
        \alpha_{2} &=& \scriptstyle \frac{-1}{12} \nu + \frac{1}{2} \nu^{2}
    \end{array}
    \right.,
\end{equation*}

\begin{equation*}
    avec \quad
    \begin{array}{rcl}
        \nu &=& \Delta t/\Delta x^{2} \\
        \nu^{2} &=& \Delta t^{2}/\Delta x^{4}
    \end{array}
\end{equation*}


\section{Consistence du schéma}

On définit l'erreur de troncature par :

\begin{equation*}
    r^{t}_{j} = u(x_{j}, t^{n+1}) -\alpha_{0} u(x_{j}, t^{n}) -\alpha_{-1} u(x_{j-1}, t^{n}) -\alpha_{-2} u(x_{j-2}, t^{n}) -\alpha_{+1} u(x_{j+1}, t^{n}) -\alpha_{+2} u(x_{j+2}, t^{n})
\end{equation*}

Par développement de Taylor du schéma (\ref{scheme}) on a:

\begin{equation*}
    u(x_{j-1}, t^{n}) =
    u(x_{j}, t^{n})
     - \Delta x \frac{\partial}{\partial x}u(x_{j}, t^{n})
     + \frac{(\Delta x)^{2}}{2} \frac{\partial^{2}}{\partial x^{2}}u(x_{j}, t^{n})
     - \frac{(\Delta x)^{3}}{3!} \frac{\partial^{3}}{\partial x^{3}}u(x_{j}, t^{n})
     + \mathcal{O}((\Delta x)^{4})
\end{equation*}

\begin{equation*}
    u(x_{j+1}, t^{n}) =
    u(x_{j}, t^{n})
     - \Delta x \frac{\partial}{\partial x}u(x_{j}, t^{n})
     + \frac{(\Delta x)^{2}}{2} \frac{\partial^{2}}{\partial x^{2}}u(x_{j}, t^{n})
     - \frac{(\Delta x)^{3}}{3!} \frac{\partial^{3}}{\partial x^{3}}u(x_{j}, t^{n})
     + \mathcal{O}((\Delta x)^{4})
\end{equation*}

\begin{equation*}
    u(x_{j-2}, t^{n}) =
    u(x_{j}, t^{n})
     - 2\Delta x \frac{\partial}{\partial x}u(x_{j}, t^{n})
     + \frac{(2\Delta x)^{2}}{2} \frac{\partial^{2}}{\partial x^{2}}u(x_{j}, t^{n})
     - \frac{(2\Delta x)^{3}}{6} \frac{\partial^{3}}{\partial x^{3}}u(x_{j}, t^{n})
     + \mathcal{O}((\Delta x)^{4})
\end{equation*}

\begin{equation*}
    u(x_{j+2}, t^{n}) =
    u(x_{j}, t^{n})
     - 2\Delta x \frac{\partial}{\partial x}u(x_{j}, t^{n})
     + \frac{(2\Delta x)^{2}}{2} \frac{\partial^{2}}{\partial x^{2}}u(x_{j}, t^{n})
     - \frac{(2\Delta x)^{3}}{6} \frac{\partial^{3}}{\partial x^{3}}u(x_{j}, t^{n})
     + \mathcal{O}((\Delta x)^{4})
\end{equation*}

\begin{equation*}
    u(x_{j}, t^{n+1}) =
    u(x_{j}, t^{n})    
     + \Delta t \frac{\partial}{\partial t}u(x_{j}, t^{n})
     + \mathcal{O}((\Delta t)^{2})
\end{equation*}

On obtient :

\begin{align*}
    & \frac{u(x_{j}, t^{n+1}) - u(x_{j}, t^{n})}{\Delta t} - \\
    & -\frac{4}{3} \frac{u(x_{j+1}, t^{n}) - 2u(x_{j}, t^{n}) + u(x_{j-1}, t^{n})}{\Delta x^{2}} + \\
    & +\frac{1}{12} \frac{u(x_{j+2}, t^{n}) - 2u(x_{j}, t^{n}) + u(x_{j-2}, t^{n})}{\Delta x^{2}} - \\
    & -\frac{\Delta t^{2}}{2} \frac{u(x_{j+2}, t^{n}) - 4u(x_{j+1}, t^{n}) + 6u(x_{j}, t^{n})  - 4u(x_{j-1}, t^{n}) + u(x_{j-2}, t^{n})}{\Delta x^{4}} = \\
   & = \mathcal{O}((\Delta t)^{2} + (\Delta x)^{4})
\end{align*}

Alors le schéma consistant et d'ordre 2 en temps et d'ordre 4 en espace.

\subsection{Stabilité}
La stabilité au sens de Von Neumann est une condition suffisante et nécessaire pour la stabilité uniforme en norme quadratique

On obtient le résultat de convergence ???

\subsection{}


\end{document}
